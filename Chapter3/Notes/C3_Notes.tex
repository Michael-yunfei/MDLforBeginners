\documentclass[12pt]{article}

% add some essential packages, some might not be used
%\usepackage{CJKutf8}
\usepackage[T1]{fontenc}
\usepackage[utf8]{inputenc}
\usepackage[usenames,dvipsnames]{color}
\usepackage{authblk}
\usepackage{xeCJK}
\usepackage{ragged2e}
\usepackage{amsmath}
%\usepackage[a4paper,margin=2in,bottom=1.0in]{geometry}
\usepackage{url}
\usepackage{array}
\usepackage{bbding}
\usepackage{amssymb}
\usepackage{graphicx}
\usepackage{adjustbox}
\usepackage{subcaption}
\usepackage{booktabs}
\usepackage{float}
\usepackage{appendix}
\usepackage{url}
\usepackage[english]{babel}
\usepackage{adjustbox}
\usepackage{textgreek}
\usepackage{NotesTeX}
\usepackage{lipsum}
\usepackage[procnames]{listings}
\usepackage{wasysym}
\usepackage{amsthm}
\usepackage{framed}
\usepackage[procnames]{listings}
\usepackage[scaled=0.9]{DejaVuSansMono}
\usepackage{pythonhighlight}

\usepackage{rotating} % for the horizontal page table

\usepackage{tikz}
\usetikzlibrary{calc}
\usetikzlibrary{matrix}
\usetikzlibrary{positioning}
\usepackage{color}
\usepackage{setspace}
% python highlight package



\usepackage{tcolorbox} % package for making colorful box

 \setlength{\parskip}{0.15cm} % change the paragraph spacing
\renewcommand\labelitemi{$\vcenter{\hbox{\tiny$\bullet$}}$} % set the bullet size as tiny

% \newcommand*\rot{\rotatebox{90}} % for rotate text

%\usepackage{courier}





\title{{\Huge 人工智能基础(高中版)}\\{\Large{辅导讲义}}}
\author{王斐 Michael\footnote{\href{https://www.michaelyunfei.com}{\textit{Author Website}}}}

\affiliation{School of Mathematics and Statistics, UCD}
\emailAdd{michael.yunfei@gmail.com}


\numberwithin{equation}{section}



\begin{document}
  \maketitle
  \flushbottom
  \newpage
  \pagestyle{fancynotes}


\setcounter{part}{2}

\part*{第三章:神经网络模型初探}

\marginnote{\begin{tcolorbox}《大学》中语:知止而后有定,定而后能静,静而后能安,安而后能虑,虑而后能得。物有本末,事有终始。知所先后,则近道矣。\end{tcolorbox}}

亲爱的同学们,从这一章开始我们将会接触目前比较流行的人工智能模型-神经网络模型。相较前一章节,本章难度有所提升,而且模型背后的相关概念更加抽象。如果你已经开始阅读第三章节的内容,或许你会觉得\textit{`不知所云'}或者\textit{`无从下手'}。 对于任何一个初学者来说,这都是很正常的经历,希望你们不要气馁,更不要怀疑自己。此时此刻,希望同学们仍然要怀着探索的心态去进入这一章节的学习。

我们将会在接下来的三周里沉浸在神经网络模型中,如果同学们紧跟老师的节奏,按照要求完成\textbf{课前预习,课上笔记,及课后练习}这三个环节,我可以向你们保证,三周之后你们完全可以:
\begin{itemize}
	\item 理解为什么人工智能在神经网络\sn{Neutral Network}出现后具有了广泛得应用价值;
	\item 掌握神经网络的本质所在,并且了解模型背后的思想精髓所在;
	\item 能够用Python透过向量\sn{Vector}和矩阵\sn{Matrix}进行编写20行左右的小程序,从而以此去理解神经网络的模型设计逻辑;
	\item 能够使用Python中人工智能学习平台,如TensorFlow来进行神经网络的训练和调试,从而可以独立自主得对大量的图片数据进行分析。
\end{itemize}

为了提高沟通速率且帮助同学们养成良好学习节奏,我们以后会将课前预习,课上笔记,及课后练习简称为C1, C2, C3。比如,如果我说\textit{你需要在下周一之前完成C1,那就意味着你需要根据我在C1中的指示进行课前预习}。下面的表格是后面课程中C1, C2, C3中的包含的内容以及所设定的预订目标。

\begin{table}[H]
	\centering
	\renewcommand{\arraystretch}{1.3}
	\begin{tabular}{ccc}
		\hline
		\hline
		C1 & C2 & C3 \\
		\hline 
		背景资料阅览 & 课堂老师讲解 & 课后习题和编程 \\
		\hline 
	\end{tabular}
\end{table}


\setcounter{section}{3}
\subsection{前言}




















\end{document}
